%----------------------------------------------------------------------------
\documentclass{article}
%----------------------------------------------------------------------------
\usepackage[a4paper, top=20mm, bottom=20mm, left=20mm, right=20mm]{geometry}
%----------------------------------------------------------------------------
\usepackage{paracol}
\usepackage{graphicx}
\usepackage{hyperref}
\usepackage[dvipsnames]{xcolor}
\usepackage{lipsum}
\usepackage{fancyhdr}
\usepackage[ngerman]{babel}
\usepackage{parskip}
\usepackage{pdflscape}
%----------------------------------------------------------------------------
\graphicspath{ {./images/} }
%----------------------------------------------------------------------------
% fancyhdr setup
%----------------------------------------------------------------------------
\pagestyle{fancy}	% activate fancy style
\fancyhf{}			% clear fancy settings
% Header
\fancyhead[L]{Datenbanksysteme: Übungsprojekt}
\fancyhead[C]{Konzeptionelles Design}
\fancyhead[R]{v1.0}
\setlength{\headheight}{13pt}
% Footer
\fancyfoot[C]{Seite \thepage}
\fancyfoot[R]{\today}
% lines
\renewcommand{\headrulewidth}{0.4pt}
\renewcommand{\footrulewidth}{0.4pt}
%----------------------------------------------------------------------------
% first page setup
%----------------------------------------------------------------------------
\title{Meilenstein 1: Konzeptionelles Design}
\author{Gruppe 30\\Boiko Bohdana, Alexander Grath}
\date{} % empty date
%----------------------------------------------------------------------------
% reference definitions (examples)
%----------------------------------------------------------------------------
\begin{filecontents}{references.bib}
@article{doe2023,
  author  = {Doe, John and Smith, Jane},
  title   = {An Analysis of Citation Styles},
  journal = {IEEE Transactions on Automatic Formatting},
  volume  = {10},
  number  = {2},
  pages   = {100--105},
  year    = {2023}
}

@book{knuth1984,
  author    = {Knuth, Donald E.},
  title     = {The TeXbook},
  publisher = {Addison-Wesley},
  year      = {1984}
}
\end{filecontents}
%----------------------------------------------------------------------------
\begin{document}
%
\maketitle
%
\thispagestyle{empty} % clear page number
%
\begin{center}
\section*{Realitätsausschnitt und konzeptueller Entwurf}
\end{center}
%
% show university logo
\begin{figure}
	\centering
	\includegraphics[scale=.6]{univie}
\end{figure}
%
\newpage
%
%\tableofcontents%(enable for table of contents)
%
\setcounter{page}{1} % restart page count at 1
%
%\newpage %(enable for table of contents)
%
% ### EXAMPLE OF USING LIPSUM AND CITATIONS ###
%\section{Kurze Beschreibung des Themengebietes und Relevanz der Studie}
%
%\lipsum[1-9]  %~1000W
%\lipsum[1-13] %~1500W
%
%This is a citation for the first paper \cite{doe2023}. 
%Here is a citation for a book \cite{knuth1984}.
% #############################################
%
\begin{paracol}{2}[\section{Definition des Realitätsausschnittes}]
%
\raggedright % fix styling for paracol
%
Aus der Angabe wurde der Vorschlag Fahrplansystem gewählt. Als reale Vorlage dient Wien Praterstern.

\medskip
Wir haben also ein System mit \textbf{Fahrzeugen}, \textbf{Personal}, ein \textbf{Streckennetz}, einen \textbf{Vertrieb} und einen \textbf{Betrieb}.

\medskip
\textbf{Fahrzeuge} haben eine \underline{\textit{Identifizierungsnummer}} ein \textit{Baujahr} und eine \textit{Kapazität}.\\ Es gib folgende Fahrzeuge:
\begin{itemize}
	\item \textbf{Busse}: haben ein \textit{Kennzeichen} und können \textit{Niederflur}busse sein
	\item \textbf{Züge}: haben eine \textit{Baureihe} und eine \textit{Wagonanzahl}
	\item \textbf{Straßenbahnen}: haben eine \textit{Spurweite} (in Wien Normalspur) und ein \textit{Stromsystem} (in Wien Gleichstrom mit Oberleitungs-Stromabnehmer)
\end{itemize}
Fahrzeuge müssen auch aufwendig gewartet werden, jede \textbf{Wartung} sorgt für mögliche Verzögerungen und sollte daher genau dokumentiert werden. Es werden eine \underline{\textit{laufende Nummer}}, das \textit{Datum}, die \textit{Kosten} und die \textit{Art der Wartung} festgehalten

\medskip
Das \textbf{Personal} mit einer \underline{\textit{Personalnummer}}, \textit{Namen}, \textit{Gehalt} und \textit{Einstellungsdatum} teil sich in unserem System in:
\begin{itemize}
	\item \textbf{Fahrer:innen}: welche eine \textit{Führerscheinklasse} haben
	\item \textbf{Kontrolleuren}: welche einen genauen \\ \textit{Zuständigkeitsbereich} haben
\end{itemize}
auf.\\
Für Ausbildungszwecke kann es vorkommen das eine Fahrer:in mehrere andere Fahrer:innen \textit{betreut}.

\medskip
Im Streckennetz gibt es:
\begin{itemize}
	\item \textbf{Lininen}: mit einer \underline{\textit{Linien Nummer}}, einer \textit{Taktung} (in Minuten), dem \textit{Beginn} und dem \textit{Ende} der \textit{Betriebszeit}
	\item \textbf{Haltestellen}: mit einer \underline{\textit{Identifikation}}, einem \textit{Namen} und möglicher \textit{Barrierefreiheit}.
\end{itemize}
Damit lässt sich ein Streckenplan erstellen der sowohl für das Personal als auch die Kunden zu verwenden ist.

\medskip
Im Vertrieb/Betrieb muss man 
\begin{itemize}
	\item \textbf{Kunden}: mit \underline{\textit{Kunden ID}}, \textit{Namen}, \textit{e-mail} und \textit{Adresse} \\ und
	\item \textbf{Tickets}: mit \underline{\textit{Ticket ID}} , \textit{Kaufdatum}, \textit{Preis} und \textit{Gültigkeitszeitraum}
\end{itemize}
abhandeln.
\switchcolumn
Die Bezahlungen werden mit den, eindeutig indentifierzierbaren, Kunden über externe Zahlungsunternehmen durchgeführt und werden nicht von diesem System erfasst.
%
\subsection*{Beziehungen der Elemente}
%
\begin{itemize}
	\item Bus, Zug und die Straßenbahn sind alles Fahrzeuge.
	\item Alle Fahrzeuge können von Wartungen betroffen sein.
	\item Fahrer:innen und Kontrolleure sind Teil des Personals.
	\item Fahrer:innen können als Mentor neue Fahrer:innen trainieren.
	\item Im Einsatz im Fahrplansystem sind sowohl Fahrer:innen mit dem Zugeordneten Fahrzeug auf der richtigen Linie.
	\item Diese Linien bedienen Haltestellen.
	\item Um fahren zu können, kaufen Kunden ein Ticket. 
	\item Dieses Ticket gilt für gewisse Linien und 
	\item werden von einem Kontrolleur geprüft.
\end{itemize}
%
\end{paracol}
%
\newpage
%
\section{Konzeptueller Entwurf}

\bigskip
%
\includegraphics[scale=0.4]{erd}

\medskip
ER-Diagram in Chen-Notation. Um die ternäre Beziehung hervorzuheben wurde n : m : p gewählt.
%
%\newpage %(enable for citations)
%
%\bibliographystyle{ieeetr} %(enable for citations)
%
%\bibliography{references} %(enable for citations)
%
\end{document}