%----------------------------------------------------------------------------
\documentclass{article}
%----------------------------------------------------------------------------
\usepackage[a4paper, top=20mm, bottom=20mm, left=20mm, right=20mm]{geometry}
%----------------------------------------------------------------------------
\usepackage{paracol}
\usepackage{graphicx}
\usepackage{hyperref}
\usepackage[dvipsnames]{xcolor}
\usepackage{lipsum}
\usepackage{fancyhdr}
\usepackage[ngerman]{babel}
\usepackage{parskip}
\usepackage{pdflscape}
%----------------------------------------------------------------------------
\graphicspath{ {./images/} }
%----------------------------------------------------------------------------
% fancyhdr setup
%----------------------------------------------------------------------------
\pagestyle{fancy}	% activate fancy style
\fancyhf{}			% clear fancy settings
% Header
\fancyhead[L]{Datenbanksysteme: Übungsprojekt}
\fancyhead[C]{Logischer und Physischer Entwurf}
\fancyhead[R]{v1.0}
\setlength{\headheight}{13pt}
% Footer
\fancyfoot[C]{Seite \thepage}
\fancyfoot[R]{\today}
% lines
\renewcommand{\headrulewidth}{0.4pt}
\renewcommand{\footrulewidth}{0.4pt}
%----------------------------------------------------------------------------
% first page setup
%----------------------------------------------------------------------------
\title{Meilenstein 3: Logischer und Physischer Entwurf}
\author{Gruppe 30\\Boiko Bohdana, Alexander Grath}
\date{} % empty date
%----------------------------------------------------------------------------
% reference definitions (examples)
%----------------------------------------------------------------------------
\begin{filecontents}{references.bib}
@article{doe2023,
  author  = {Doe, John and Smith, Jane},
  title   = {An Analysis of Citation Styles},
  journal = {IEEE Transactions on Automatic Formatting},
  volume  = {10},
  number  = {2},
  pages   = {100--105},
  year    = {2023}
}

@book{knuth1984,
  author    = {Knuth, Donald E.},
  title     = {The TeXbook},
  publisher = {Addison-Wesley},
  year      = {1984}
}
\end{filecontents}
%----------------------------------------------------------------------------
\begin{document}
%
\maketitle
%
\thispagestyle{empty} % clear page number
%
\begin{center}
\section*{Logischer und Physischer Entwurf}
\end{center}
%
% show university logo
\begin{figure}
	\centering
	\includegraphics[scale=.6]{univie}
\end{figure}
%
\newpage
%
%\tableofcontents%(enable for table of contents)
%
\setcounter{page}{1} % restart page count at 1
%
%\newpage %(enable for table of contents)
%
% ### EXAMPLE OF USING LIPSUM AND CITATIONS ###
%\section{Kurze Beschreibung des Themengebietes und Relevanz der Studie}
%
%\lipsum[1-9]  %~1000W
%\lipsum[1-13] %~1500W
%
%This is a citation for the first paper \cite{doe2023}. 
%Here is a citation for a book \cite{knuth1984}.
% #############################################
%
\section{Normalisierung (3. Normalform)}
\begin{paracol}{2}[\section{Section}]
%
\switchcolumn
%
\end{paracol}
%
%\newpage %(enable for citations)
%
%\bibliographystyle{ieeetr} %(enable for citations)
%
%\bibliography{references} %(enable for citations)
%
\end{document}